\section{Perintah Navigasi}
Perintah navigasi direktori

\section{Yusniar Nur Syarif Sidiq\_1164089}
\section{Teori}

\begin{enumerate}

\item Python memiliki dua Vaiabel yaitu integer dan string
\begin{itemize}
	\item Contoh Variabel String
\end{itemize}
	\begin{verbatim}
		x="YusniarNS"
		y="D4 Teknik Informatika"
		z=x + y
		print(z)
	\end{verbatim}
	\subitem Output dari source code tersebut adalah YusniarNS D4 Teknik Informatika
	\item Contoh Variabel Integer
	\begin{verbatim}
		x=5
		y=10
		print(x+y)
	\end{verbatim}
	\subitem Output dari source code tersebut adalah 15

\par
\item Cara melakukan proses input
	\begin{verbatim}
		print("Enter your name:")
		x= input()
		print("Hello, "+x)
	\end{verbatim}
\subitem Dimana ketika kita running akan meminta form inputan dan apabila saya berikan inputan YusniarNS maka output yang keluar adalah Hello YusniarNS di karenakan variabel x adalah inputan yang kita berikan.
\par
 \begin{itemize}
	\item Aritmatika Pertambahan
\end{itemize}
		\begin{verbatim}
			x=5
			y=3
			print(x+y)
		\end{verbatim}
	\subitem Ouput yang keluar dari source code tersebut adalah 8 di karenakan 5 ditambah 3 sama dengan 8.
	
	\item Aritmatika Pengurangan
		\begin{verbatim}
			x=5
			y=3
			print(x-y)
		\end{verbatim}
	\subitem Output yang keluar dari source code tersebut adalah 2.

	\item Aritmatika Pengurangan
		\begin{verbatim}
			x=5
			y=3
			print(x*y)
		\end{verbatim}
	\subitem Output yang keluar dari source code tersebut adalah 15.

	\item Aritmatika Pembagian
		\begin{verbatim}
			x=5
			y=3
			print(x/y)
		\end{verbatim}
	\subitem Output yang keluar dari source code tersebut adalah 1,67.

	\item Merubah Integer Ke String
		\begin{verbatim}
			x=5
			y=3
			z=str(x)+str(y)
		\end{verbatim}

	\item Merubah String Ke Integer
		\begin{verbatim}
			x='5'
			y='3'
			z=int(x)+int(y)
		\end{verbatim}
\par
\item Python memiliki dua type pengulangan yaitu While Looping dan For Looping

\begin{itemize}
	\item Contoh While Looping
\end{itemize}
		\begin{verbatim}
			i = 1
			while i < 6:
		  	print(i)
 			 i += 1
		\end{verbatim}
	\subitem Output yang akan keluar adalah memunculkan angka 1 samapi 5 dikarenakan adanya source code
		\begin{verbatim}
			while i<6
		\end{verbatim}
	
	\item Contoh For Looping
		\begin{verbatim}
			for x in "banana":
			print(x)
		\end{verbatim}
	\subitem Output yang keluar adalah mengulang huruf b a n a n a secara vertikal.
\par
\item Contoh source code condition adalah seperti berikut
	\begin{verbatim}
		a = 33
		b = 33
		if b > a:
  		print("b is greater than a")
		elif a == b:
  		print("a and b are equal")
	\end{verbatim}
\subitem Output dari source code tersebut adalah a and b are equal di karenakan nilai vaiabel a dan b sama.

\par
\item Jenis Erorr yang ditemukan
\subitem dalam kasus ini erorr yang saya temukan adalah source code pada nomor aritmatika pembagian.
	\begin{verbatim}
		x=5
		y='3'
		print(x/y)
	\end{verbatim}
\subitem pada source code tersebut akan terjadi erorr dikarenakan integer dan string tidak dapat di persatukan. Untu penyelesaiannya adalah sebagai berikut.
	\begin{verbatim}
		x=5
		y=3
		print(x/y)
	\end{verbatim}
\par
	\begin{verbatim}
		x = 3
		try:
		print(x)
		except NameError:
		print("Variable x is not defined")
		except:
		print("Something else went wrong")
	\end{verbatim}
\subitem Output yang dikeluarkan adalah 3 dikarenakan variabel dari x bernilai 3.
\end{enumerate}


