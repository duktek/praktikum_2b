%%%%%%%%%%%%%%
%% Run LaTeX on this file several times to get Table of Contents,
%% cross-references, and citations.

%% If you have font problems, you may edit the w-bookps.sty file
%% to customize the font names to match those on your system.

%% w-bksamp.tex. Current Version: Feb 16, 2012
%%%%%%%%%%%%%%%%%%%%%%%%%%%%%%%%%%%%%%%%%%%%%%%%%%%%%%%%%%%%%%%%
%
%  Sample file for
%  Wiley Book Style, Design No.: SD 001B, 7x10
%  Wiley Book Style, Design No.: SD 004B, 6x9
%
%
%  Prepared by Amy Hendrickson, TeXnology Inc.
%  http://www.texnology.com
%%%%%%%%%%%%%%%%%%%%%%%%%%%%%%%%%%%%%%%%%%%%%%%%%%%%%%%%%%%%%%%%

%%%%%%%%%%%%%
% 7x10
%\documentclass{wileySev}

% 6x9
\documentclass{wileySix}

\usepackage{graphicx}
\usepackage{listings}

\usepackage{color}
 
\definecolor{codegreen}{rgb}{0,0.6,0}
\definecolor{codegray}{rgb}{0.5,0.5,0.5}
\definecolor{codepurple}{rgb}{0.58,0,0.82}
\definecolor{backcolour}{rgb}{0.95,0.95,0.92}
 
\lstdefinestyle{mystyle}{
    backgroundcolor=\color{backcolour},   
    commentstyle=\color{codegreen},
    keywordstyle=\color{magenta},
    numberstyle=\tiny\color{codegray},
    stringstyle=\color{codepurple},
    basicstyle=\footnotesize,
    breakatwhitespace=false,         
    breaklines=true,                 
    captionpos=b,                    
    keepspaces=true,                 
    numbers=left,                    
    numbersep=5pt,                  
    showspaces=false,                
    showstringspaces=false,
    showtabs=false,                  
    tabsize=2,
    language=sh
}
 
\lstset{style=mystyle}

%%%%%%%
%% for times math: However, this package disables bold math (!)
%% \mathbf{x} will still work, but you will not have bold math
%% in section heads or chapter titles. If you don't use math
%% in those environments, mathptmx might be a good choice.

% \usepackage{mathptmx}

% For PostScript text
\usepackage{w-bookps}

%%%%%%%%%%%%%%%%%%%%%%%%%%%%%%%%%%%%%%%%%%%%%%%%%%%%%%%%%%%%%%%%
%% Other packages you might want to use:

% for chapter bibliography made with BibTeX
% \usepackage{chapterbib}

% for multiple indices
% \usepackage{multind}

% for answers to problems
% \usepackage{answers}

%%%%%%%%%%%%%%%%%%%%%%%%%%%%%%
%% Change options here if you want:
%%
%% How many levels of section head would you like numbered?
%% 0= no section numbers, 1= section, 2= subsection, 3= subsubsection
%%==>>
\setcounter{secnumdepth}{3}

%% How many levels of section head would you like to appear in the
%% Table of Contents?
%% 0= chapter titles, 1= section titles, 2= subsection titles, 
%% 3= subsubsection titles.
%%==>>
\setcounter{tocdepth}{2}

%% Cropmarks? good for final page makeup
%% \docropmarks

%%%%%%%%%%%%%%%%%%%%%%%%%%%%%%
%
% DRAFT
%
% Uncomment to get double spacing between lines, current date and time
% printed at bottom of page.
% \draft
% (If you want to keep tables from becoming double spaced also uncomment
% this):
% \renewcommand{\arraystretch}{0.6}
%%%%%%%%%%%%%%%%%%%%%%%%%%%%%%

%%%%%%% Demo of section head containing sample macro:
%% To get a macro to expand correctly in a section head, with upper and
%% lower case math, put the definition and set the box 
%% before \begin{document}, so that when it appears in the 
%% table of contents it will also work:

\newcommand{\VT}[1]{\ensuremath{{V_{T#1}}}}

%% use a box to expand the macro before we put it into the section head:

\newbox\sectsavebox
\setbox\sectsavebox=\hbox{\boldmath\VT{xyz}}

%%%%%%%%%%%%%%%%% End Demo


\begin{document}


\booktitle{Cerdas Menguasai Python}
\subtitle{Dalam 24 Jam}

\authors{Rolly M. Awangga\\
\affil{Informatics Research Center}
%Floyd J. Fowler, Jr.\\
%\affil{University of New Mexico}
}

\offprintinfo{Cerdas Menguasai Python, First Edition}{Rolly M. Awangga}

%% Can use \\ if title, and edition are too wide, ie,
%% \offprintinfo{Survey Methodology,\\ Second Edition}{Robert M. Groves}

%%%%%%%%%%%%%%%%%%%%%%%%%%%%%%
%% 
\halftitlepage

\titlepage


\begin{copyrightpage}{2019}
\input{info/copyrightpage}
\end{copyrightpage}

\dedication{`Jika Kamu tidak dapat menahan lelahnya belajar, 
Maka kamu harus sanggup menahan perihnya Kebodohan.'
~Imam Syafi'i~}

\begin{contributors}
\input{info/contributors}
\end{contributors}

\contentsinbrief
\tableofcontents
\listoffigures
\listoftables
\lstlistoflistings


\begin{foreword}
\input{info/foreword}
\end{foreword}

\begin{preface}
\input{info/preface}
\end{preface}


\begin{acknowledgments}
\input{info/acknowledgments}
\end{acknowledgments}

\begin{acronyms}
\input{info/acronyms}
\end{acronyms}

\begin{glossary}
\input{info/glossary}
\end{glossary}

\begin{symbols}
\input{info/symbols}
\end{symbols}

\begin{introduction}
\input{info/introduction}
\end{introduction}

%%%%%%%%%%%%%%%%%%Isi Buku_

\chapter{Judul Bagian Pertama}
\section{Irvan Rizkiansyah}

\section{Python}
	\subsection{Background}
	\label{Background}
	\par
	Python adalah sebuah bahasa pemrograman yang bersifat interpreter, interactive, object-oriented, dan dapat beroperasi hampir pada semua platform seperti Windows, Linux, Mac. Python termasuk sebagai bahasa pemrograman yang dapat dengan mudah di pelajari karena sintaks yang jelas dan mudah dipahami, dan dapat dikombinasikan dengan penggunaan modul yang siap pakai, dan struktur data tingkat tinggi yang efisien \cite{prasetya2012deteksi}.
	\par
	Python memiliki kepustakaan atau biasa disebut library yang sangat luas, dan dalam distribusi Python yang telah disediakan, hal tersebut diakibatkan oleh pendistribusian Python yang bebas karena bahasa pemrograman Python merupakan bahasa pemrograman yang freeware atau bebas dalam hal pengembangannya. Python adalah sebuah bahasa pemrograman yang dapat dengan mudah dibaca dan terstruktur, hal tersebut dikarenakan penggunaan sistem identasi, yaitu pemisahan blok-blok program susunan identasi, jadi untuk menambahkan sub-sub program dalam sebuah blok program, sub program tersebut harus diletakkan pada satu atau lebih spasi dari kolom sebuah blok \cite{perkasa2014rancang}.
	\par
	Bahasa pemrograman Python dibuat oleh Guido Van Rossum. Dikarenakan para pengembang software atau perangkat lunak lebih cenderung memilih kecepatan dalam menyelesaikan suatu proyek dibandingkan dengan kecepatam proses dari program yang dijalankan, maka dari itu bahasa pemrograman Python dapat dibilang bahasa pemerograman yang kecepatannya dapat melebihi bahasa pemrograman C. Akan tetapi bahasa pemrograman Python lebih lambat dalam memproses suatu program dibandingkan bahasa pemrograman C. dengan berkembangnya kecepatan prosesor dan memori saat ini, mengakibatkan tidak terlihatnya keterlambatan dari sebuah program yang menggunakan bahasa pemrograman Python \cite{miftakhuddinimplementasi}.

	\subsection{Problems}
		\begin{itemize}
			\item Kurangnya pemahaman tentang bahasa pemrograman Python
			\item Kurang mengerti dalam hal fungsi-fungsi yang terdapat pada bahasa pemrograman Python
		\end{itemize}

	\subsection{Objective and Contribution}
		\subsubsection{Objective}
			\begin{itemize}
				\item Dapat memahami tentang bahasa pemrograman Python
				\item Dapat memahami fungsi fungsi yang terdapat pada bahasa pemrograman Python
			\end{itemize}
	
		\subsubsection{Contribution}
			\begin{itemize}
				\item Dapat membangun sebuah sistem dengan menggunakan bahasa pemrograman Python
				\item Dapat membangun sebuah alat yang berguna, menggunakan mikrokontroler dan bahasa pemrograman python
			\end{itemize}

	\subsection{Scoop and Environtment}
		\begin{itemize}
			\item Pengenanalan tentang bahasa pemrograman Python
			\item Pengenalan fungsi-fungsi yang terdapat pada bahasa pemrograman Python
		\end{itemize}

\section{Luthfi Muhammad Nabil\_1174035}
\subsection{Background}
Python adalah sebuah bahasa pemrograman dengan level tinggi yang interaktif, dan mendukung berbagai paradigma pemrograman. Python sudah terkenal pada kalangan programmer sebagai bahasa yang mudah dipahami dan memiliki kompleksitas yang dinamis sehingga dapat dipakai di algoritma maupun platform yang berbagai macam.Python sudah memiliki banyak komunitas pendukung karena penggunanya yang banyak. Selain pada komunitas biasa, Python sudah diimplementasikan pada banyak perusahaan ternama dan dipasang pada aplikasi yang sudah terkenal seperti pada search engine google yang dimiliki oleh perusahaan Google. 
\linebreak
\linebreak
Python mulai dirilis pada tahun 1991 oleh Guido van Rossum sebagai kelanjutan dari bahasa pemrograman ABC dengan memiliki versi yaitu 0.9.0. Nama dari bahasa Python diambil dari program televisi di Inggris bernama Monty Python. Lalu tahun 1995, Guido pindah ke CNRI di Virginia, Amerika sembari melanjutkan pengembangan Python. Versi terakhir yang dikeluarkan telah mencapai 1.6. Pada awalnya, Python adalah bahasa yang dipakai untuk  Lalu pada tahun 2000, dirilis Python versi 2.0 yang memiliki peran sebagai bahasa pemrograman tidak berbayar atau open source. Van Rossum sendiri aktif pada development dari Python tetapi sudah bergabung dengan banyak penyumbang. Dibandingkan dengan bahasa lain, Python sudah melewati beberapa versi yang terbatas, mengikuti filosofi dari perubahan berurutan. 
\linebreak
\linebreak
Untuk memahami bahasa Python tidak sulit, tetapi instalasi Python cukup memiliki trik tersendiri terlebih untuk pengguna yang baru memasuki lingkup programming. Pada sistem operasi windows, pengguna diharuskan untuk memasuki sistem pada windows untuk mengatur lokasi dari Python yang sudah diinstall. Selain itu, untuk yang terbiasa dengan beberapa pemrograman harus beradaptasi dengan aturan - aturan pada bahasa pemrograman Python seperti penggantian titik koma (;) dengan indentasi. Oleh karena itu, penulis akan membahas mengenai pengenalan singkat mengenai bahasa pemrograman python dan cara instalasi dari python dan library pip.

\subsection{Problems}
Sesuai dengan latar belakang yang telah dibahas, penulis merumuskan masalah sebagai berikut : 
\begin{enumerate}	
	\item Bagaimana pemaparan singkat mengenai Python?
	\item Bagaimana cara melakukan instalasi Python?
\end{enumerate}

\subsection{Objective and Contribution}
\subsubsection{Objective}
\begin{enumerate}
	\item Untuk membahas mengenai Python.
	\item Untuk menunjukkan cara instalasi Python.
\end{enumerate}

\subsubsection{Contribution}
Pada materi ini, penulis menggunakan Python.

\subsection{Scoop and Environment}
\begin{itemize}
	\item Pada Chapter 1 membahas mengenai sejarah, latar belakang, dan keterangan singkat mengenai python tersebut. Chapter ini juga merangkum masalah dan mencari tujuan yang ingin dicapai penulis dalam membuat resume ini.
\end{itemize}

\section{Hagan Rowlenstino/1174040}
\subsection{Background}
Python di desain sebagai bahasa pemrograman yang dapat digunakan sehari-hari. Pencipta python ,Guido van Rossum, telah menulis seri lengkap tentang sejarah bahasa tersebut.Python diciptakan di awal 1990 di CWI \'(the Centrum voor Wiskunde and Informatica), tempat kelahiran ALGOL \'(Algorithmic Language 68 ). Sebelumnya, Rossum juga telah mengerjakan bahasa pemrograman ABC, yang dikembangkan di  CWI sebagai bahasa pengajaran yang menekankan kejelasan. Walaupun project ABC telah di tutup , Rossum banyak belajar dari hal tersebut saat dia mulai membuat Python sebagai alat untuk multimedia dan project penelitian sistem operasi. Dia ingin Python mempunyai tingkatan yang cukup tinggi agar mudah untuk dibaca dan ditulis, juga mirip dengan Java, dan menawarkan portabilitas serta error model yang terdefinisi dengan baik.
\linebreak
\linebreak
Python juga kaya akan vocabulary yang berguna untuk membuat algoritma yang kompleks dengan efisien dikarenakan punya dictionaries yang memiliki string yang kuat dan assosiasi array yang fleksibel. Python menggabungkan antara fleksibilitas tingkat tinggi, kemampuan membaca, dan interface yang terdefinisi dengan baik. Kombinasi tersebut membuat Python cocok untuk menyelesaikan masalah komputasi non-algoritma seperti integrase dengan web, format data, ataw hardware kelas rendah. Python mudah untuk dipelajari karena strukturnya sederhana dan sintaksnya jelas, punya library yang portable dan dapat digunakan di beda perangkat,dan dapat terintegrasi dengan bahasa pemrograman lain seperti C, C++, dan Java.
\subsection{Problems}
\begin{enumerate}
\item Banyak pemrograman yang penggunaannya kompleks
\end{enumerate}
\subsection{Objective and Contribution}
\subsubsection{Objective}
\begin{enumerate}
\item Dapat memudahkan pemrograman dengan bahasa pemrograman yang tepat
\end{enumerate}
\subsubsection{Contribution}
\begin{enumerate}
\item Menggunakan Python sebagai bahasa pemrograman
\end{enumerate}
\subsection{Scoop and Environment}
\begin{enumerate}
\item Mengimplementasikan Python dalam pemrograman
\end{enumerate}


\section{Faisal Najib Abdullah 1174042}
\subsection{Background}
\label{Background}
\par
Python lahir pada akhir tahun 1980 an dan implementasinya dimulai pada Desember 1989 oleh Guido van Rossum di CWI di Belanda sebagai penerus bahasa ABC (itu sendiri terinspirasi oleh SETL) yang mampu menangani pengecualian dan berinteraksi dengan sistem operasi Amuba. Van Rossum adalah penulis utama Python, dan peran sentralnya yang berkelanjutan dalam menentukan arah Python tercermin dalam judul yang diberikan kepadanya oleh komunitas Python, Benevolent Dictator for Life (BDFL).
\par
Python adalah bahasa pemrograman interpretatif multiguna dengan filosofi desain yang berfokus pada keterbacaan kode dan python sendiri diklaim sebagai bahasa yang menggabungkan kapabilitas, kemampuan, dengan kode sintaksis yang sangat jelas, dan dilengkapi dengan fungsi pustaka standar yang besar dan komprehensif. Python juga didukung oleh komunitas besar.
\par
Python mendukung pemrograman multi paradigma, terutama tetapi tidak terbatas pada pemrograman berorientasi objek, pemrograman imperatif, dan pemrograman fungsional. Salah satu fitur yang tersedia di Python adalah sebagai bahasa pemrograman dinamis yang dilengkapi dengan manajemen memori otomatis. Python menggunakan bahasa bahasa scripting yang sama seperti bahasa pemrograman dinamis, meskipun dalam praktiknya penggunaan bahasa ini lebih luas mencakup konteks penggunaan yang umumnya tidak dilakukan menggunakan bahasa skrip. Python dapat digunakan untuk keperluan pengembangan perangkat lunak dan dapat berjalan di berbagai platform sistem operasi.
\par
CPython, implementasi referensi Python, adalah perangkat lunak bebas dan open source dan memiliki model pengembangan berbasis komunitas, seperti halnya hampir semua implementasi alternatifnya. CPython dikelola oleh Yayasan Perangkat Lunak Python nirlaba \cite{van2007python}.

\subsection{Problems}
\begin{itemize}
	\item Mahasiswa D4 TI belum dapat belum memahami apa itu python
    \item Mahasiswa D4 TI belum mengerti fungsi fungsi apa saja yang terdapat pada python
    \item Mahasiswa D4 TI belum dapat menjalankan fungsi python
\end{itemize}

\subsection{Objective and Contribution}
\subsubsection{Objective}
\begin{itemize}
	\item Mahasiswa D4 TI dapat memahami apa itu python
	\item Mahasiswa D4 TI dapat memahami fungsi fungsi yang terdapat pada python
	\item Mahasiswa D4 TI dapat menjalankan fungsi python
\end{itemize}
	
\subsubsection{Contribution}
\begin{itemize}
	\item Mahasiswa D4 TI dapat membangun suatu aplikasi yang mengimplementasikan bahasa python
	\item Mahasiswa D4 TI dapat membangun alat yang terhubung dengan aplikasi menggunakan bahasa python
\end{itemize}

\subsection{Scoop and Environtment}
\begin{itemize}
	\item Mengenali apa itu python pada mahasiswa
	\item Mengenali fungsi fungsi dasar python dan menjalankannya
\end{itemize}


\section{Ichsan Hizman Hardy/1174034}
%cite belum di upload .bib, jadi dihapus dulu
\subsection{Background}
\par
Python merupakan bahasa pemrograman interpretatif multiguna. Python pertama kali, diciptakan oleh Guido van Rossum di Stichting Mathematisch Centrum atau CWI di Belanda pada tahun 1990. pada tahun 1995, Guido melanjutkan karyanya pada Python di Virginia, dimana ia telah meliris beberapa versi perangkat lunak.
Tidak seperti bahasa lain yang sulit dibaca dan dipahami, python menekankan keterbacaan kode untuk membuatnya lebih mudah untuk memahami sintaksis.Ini membuat Python sangat mudah dipelajari untuk pemula dan mereka yang telah menguasai bahasa pemrograman lain.
\par
Python dengan desian yang sangat mudah di baca dan dipahami, karena sama seperti bahasa pemrograman yang lainnya yaitu dengan menggunakan bahasa inggris. selain itu juga lebih sedikit dalam penggunaan rumus atau syntac.
\par
Pyton juga mendukung sistem teknik pemrograman yang merangkum kode dalam objek. Bahasa Python  mendukung hampir  semua sistem operasi, termasuk operasi Linux.
\par
Dengan kode yang simpel dan mudah diimplementasikan, seorang programer dapat lebih mengutamakan pengembangan aplikasi yang dibuat. Kamu bisa menggunakan Python untuk membuat aplikasi berbasis web, game, atau bahkan sebuah search engine
	
\subsection{Problems}
\begin{enumerate}
	\item Mahasiswa D4TI2B belum bisa menggunakan bahasa python.
	\item Bagaimana pengaruh bahasa python terhadap mahasiswa D4TI2B.
	\item Bagaimana penggunaan bahasa python terhadap web service.
\end{enumerate}
	
\subsection{Objective and Contribution}
\subsubsection{Objective}
\begin{enumerate}
	\item Mahasiswa D4TI2B mampu memahami bahasa pemrograman python secara bertahap.
	\item Bahasa pemrograman python mampu mempengaruhi mahasiswa D4TI2B menjadi lebih semangat dalam belajar web service.
	\item Penggunaan bahasa python mampu mempermudah mahasiswa dalam membuat web service.
\end{enumerate}
\subsubsection{Contribution}
\begin{enumerate}
	\item Membantu mahasiswa D4TI2B dalam menyelesaikan masalah pada python.
	\item Membantu mahasiswa D4TI2B memahami bahasa pemrograman python.
	\item Mempelajari bahasa python dalam proses pembuatan web service.
\end{enumerate}
		
\subsection{Scope and Environtment}
\begin{enumerate}
	\item Mahasiswa D4TI2B memahami bahasa pemrograman python.
	\item Mahasiswa D4TI2B mampu menjalankan fungsi python.
	\item Mahasiswa D4TI2B mampu membuat web service menggunakan python.
\end{enumerate}

\section{Kevin Natanael Nainggolan 1174059}
\subsection{Background}
\par
Python adalah bahasa pemrograman tinggkat tinggi yang interpretatif dan memiliki tujuan umum. Python dibuat oleh Guido Van Rossum dan perilisan pertamanya di tahun 1991, Python di desain untuk memberikan performa dalam membaca kode, hal ini memberikan kontruksi untuk membuat pemrograman yang jelas dalam skala besar maupun kecil. Python sendiri sudah dipahami sejak tahun 1980-an oleh Guido Van Rossum di Centrum Wiskunde and Informatic di Belanda sebagai bahasa penerus ABC. Bahasa ini dapat berinteraksi pada dengan sistem operasi Amoeba dan mulai diimplementasikan di bulan  Desember 1989.
\par
Pengembangan python ada beberapa, seperti perilisan Pyhton 2.0 menyediakan fitur  pengumpul sampah pendeteksi siklus dan dapat digunakan dalam Unicode, python 3.0 yang dirilis pada 3 Desember 2008, memberikan fitur yang memberikan banyak fitur yang dibuat untuk revisi python versi 2.6.x dan 2.7.x dan penerjemah dari kode python 2 ke kode python 3, lalu python 2.7s yang ditetapkan tahun 2015 ditunda hingga 2020 mendatang dikarenakan kekhawatiran bahwa kode yang ada sebagian besar tidak dapat di-porting ke python 3. Januari 2017, google mengumumkan pengerjaan python 2.7 untuk melakukan pengembangan lebih lanjut.

	
\subsection{Problems}
\begin{enumerate}
	\item Pemahaman yang salah tentang Logic error dan Syntax error.
	\item Sulit membedakan mana jenis error antara Logic error dan Syntax error.
\end{enumerate}
	
\subsection{Objective and Contribution}
\subsubsection{Objective}
\begin{enumerate}
	\item Orang-orang yang belum paham tentang peredaan Logic Error dan Syntax error dapat memahimi perbedaanya.
	\item Dapat menentukan dan menyelesaikan error pada Logic error maupun Syntax error.
\end{enumerate}
\subsubsection{Contribution}
\begin{enumerate}
	\item Menjelaskan definisi dari Logic error dan Syntax error.
	\item Memberikan gambaran tentang masing-masing error agar dapat membedakannya.
\end{enumerate}
		
\subsection{Scope and Environtment}
\begin{enumerate}
	\item Pemahaman Logic error dan Syntax error dalam python.
\end{enumerate}

\section{DikaSukmaPradana\_1174050}
\subsection{Background Python}
	\par 
	Python adalah bahasa pemrograman tingkat tinggi untuk keperluan umum yang filosofi desainnya menekankan keterbacaan kode. Sintaksis Python memungkinkan programmer untuk mengekspresikan konsep dalam lebih sedikit baris kode daripada yang mungkin dilakukan dalam bahasa seperti C, dan bahasa tersebut menyediakan konstruksi yang dimaksudkan untuk memungkinkan program yang jelas pada skala kecil dan besar\cite{van2007python}.
	\par
	Python mendukung banyak paradigma pemrograman, termasuk gaya pemrograman berorientasi objek, imperatif dan fungsional. Ini fitur sistem tipe yang sepenuhnya dinamis dan manajemen memori otomatis, mirip dengan Skema, Ruby, Perl dan Tclm dan memiliki perpustakaan standar yang besar dan komprehensif\cite{van2007python}.
	\par
	Seperti bahasa dinamis lainnya, Python sering digunakan sebagai bahasa scripting, tetapi juga digunakan dalam berbagai konteks non-scripting. Menggunakan alat pihak ketiga, kode Python dapat dikemas ke dalam program yang dapat dieksekusi mandiri. Penerjemah python tersedia untuk banyak sistem operasi\cite{van2007python}.
	\par
	CPython, implementasi referensi Python, adalah perangkat lunak bebas dan open source dan memiliki model pengembangan berbasis komunitas, seperti halnya hampir semua implementasi alternatifnya. CPython dikelola oleh Yayasan Perangkat Lunak Python nirlaba\cite{van2007python}.
	\par 
	Python dikandung pada akhir 1980-an dan implementasinya dimulai pada Desember 1989 oleh Guido van Rossum di CWI di Belanda sebagai penerus bahasa ABC (itu sendiri terinspirasi oleh SETL) yang mampu menangani pengecualian dan berinteraksi dengan sistem operasi Amuba. Van Rossum adalah penulis utama Python, dan peran sentralnya yang terus menerus dalam menentukan arah Python adalah komunitas Python, Diktator Kebajikan untuk Hidup (BDFL)\cite{van2007python}.
	\par 
	Python 2.0 dirilis pada 16 Oktober 2000, dengan banyak fitur baru termasuk pengumpul sampah penuh dan dukungan untuk Unicode. Dengan rilis ini, proses pengembangan diubah dan menjadi lebih transparan dan didukung masyarakat\cite{van2007python}.
	\par
	Python 3.0 (juga disebut Python 3000 atau py3k), rilis utama yang tidak kompatibel dengan versi terbelakang, dirilis pada 3 Desember 2008 setelah periode pengujian yang panjang. Banyak fitur utamanya telah di-backport ke Python 2.6 dan 2.7 yang kompatibel dengan backwards\cite{van2007python}.
	
\subsection{Problems}
	\begin{enumerate}
		\item Bagaimana mahasiswa D4TI2B bisa menggunakan bahasa python.
		\item Bagaimana pengaruh bahasa python terhadap mahasiswa D4TI2B.

\section{Muhammad Iqbal Panggabean/1174063}
\subsection{Background}
	\par 
	Python adalah salah satu bahasa pemograman tingkat tinggi yang bersifat interpreter, interactive, objectoriented, dan dapat beroperasi hampir di semua platform: Mac, Linux, dan Windows. Python termasuk bahasa pemograman yang mudah dipelajari karena sintaks yang jelas, dapat dikombinasikan dengan penggunaan modulmodul siap pakai, dan struktur data tingkat tinggi yang efisien \cite{prasetya2012deteksi}. 
	\par
	Distribusi Python dilengkapi dengan suatu fasilitas seperti shell di Linux. Lokasi penginstalan Python biasa terletak di “/usr/bin/Python”, dan bisa berbeda. Menjalankan Python, cukup dengan mengetikan “Python”, tunggu sebentar lalu muncul tampilan “>>>”, berarti Python telah siap menerima perintah. Ada juga tanda “...” yang berarti baris berikutnya dalam suatu blok prompt '>>>'. Text editor digunakan untuk modus skrip \cite{obrst2003semantic}.
	\par
	Untuk membangun penelitian ini digunakan wxPython yang merupakan toolkit GUI untuk bahasa pemrograman Python. wxPython memungkinkan programmer Python untuk membuat aplikasi dengan pondasi kuat, grafis antarmuka dengan pengguna yang sangat fungsional, sederhana, dan mudah. wxPython diimplementasikan sebagai modul ekstensi oleh Python (kode asli). wxPython membungkus wxWidget sebagai salah satu GUI library populer yang ditulis dalam bahasa C++. Selain itu, digunakan pula Boa Constructor yang merupakan Integrated Development Environment (IDE) untuk Python dan wxPython GUI Builder yang cross-platform. Boa Constructor mampu membuat, memanipulasi frame secara visual (tanpa skrip), dan ada banyak object inspector seperti: browser objek, hirarki warisan, debugger yang canggih, dan bantuan yang sudah terintegrasi \cite{malikhah2016eksplorasi}.
	
\subsection{Problems}
	\begin{enumerate}
		\item Bagaimana mahasiswa Tehnik Informatika Politeknik Pos Indonesia bisa menggunakan bahasa pemrograman python
		\item Kenapa mahasiswa Tehnik Informatika Politeknik Pos Indonesia harus belajar bahasa pemrograman python

		\item Bagaimana penggunaan bahasa python terhadap web service.
	\end{enumerate}
	
\subsection{Objective and Contribution}
	\subsubsection{Objective}
		\begin{enumerate}

			\item Mahasiswa D4TI2B mampu memahami bahasa pemrograman python secara bertahap.
			\item Bahasa pemrograman python mampu mempengaruhi mahasiswa D4TI2B menjadi lebih semangat dalam belajar web service.

			\item Mahasiswa Tehnik Informatika Politeknik Pos Indonesia mampu memahami bahasa pemrograman python secara bertahap.
			\item Bahasa pemrograman python Dapat dijalankan di Linux, Mac, Windows dan termasuk perangkat mobile.

			\item Penggunaan bahasa python mampu mempermudah mahasiswa dalam membuat web service.
		\end{enumerate}
	\subsubsection{Contribution}
		\begin{enumerate}
			\item Membantu mahasiswa D4TI2B dalam menyelesaikan masalah pada python.
			\item Membantu mahasiswa D4TI2B memahami bahasa pemrograman python.
			\item Mempelajari bahasa python dalam proses pembuatan web service.
			\item Membantu mahasiswa Tehnik Informatika dalam menyelesaikan masalah pada python.
			\item Membantu mahasiswa Tehnik Informatika memahami bahasa pemrograman python.
			\item Mempelajari bahasa Python dalam proses pembuatan web service.
		\end{enumerate}
		
\subsection{Scope and Environtment}
	\begin{enumerate}
		\item Mahasiswa D4TI2B memahami bahasa pemrograman python.
		\item Mahasiswa D4TI2B mampu menjalankan fungsi python.
		\item Mahasiswa D4TI2B mampu membuat web service menggunakan python.
	\end{enumerate}
		\item Mahasiswa Tehnik Informatika Politeknik Pos Indonesia memahami bahasa pemrograman python.
		\item Mahasiswa Tehnik Informatika Politeknik Pos Indonesia mampu menjalankan fungsi python.
		\item Mahasiswa Tehnik Informatika Politeknik Pos Indonesia mampu membuat web service menggunakan python.
	\end{enumerate}


\chapter{Judul Bagian Kedua}
\section{Perintah Navigasi}
Perintah navigasi direktori

\section{Yusniar Nur Syarif Sidiq\_1164089}
\section{Teori}

\begin{enumerate}

\item Python memiliki dua Vaiabel yaitu integer dan string
\begin{itemize}
	\item Contoh Variabel String
\end{itemize}
	\begin{verbatim}
		x="YusniarNS"
		y="D4 Teknik Informatika"
		z=x + y
		print(z)
	\end{verbatim}
	\subitem Output dari source code tersebut adalah YusniarNS D4 Teknik Informatika
	\item Contoh Variabel Integer
	\begin{verbatim}
		x=5
		y=10
		print(x+y)
	\end{verbatim}
	\subitem Output dari source code tersebut adalah 15

\par
\item Cara melakukan proses input
	\begin{verbatim}
		print("Enter your name:")
		x= input()
		print("Hello, "+x)
	\end{verbatim}
\subitem Dimana ketika kita running akan meminta form inputan dan apabila saya berikan inputan YusniarNS maka output yang keluar adalah Hello YusniarNS di karenakan variabel x adalah inputan yang kita berikan.
\par
 \begin{itemize}
	\item Aritmatika Pertambahan
\end{itemize}
		\begin{verbatim}
			x=5
			y=3
			print(x+y)
		\end{verbatim}
	\subitem Ouput yang keluar dari source code tersebut adalah 8 di karenakan 5 ditambah 3 sama dengan 8.
	
	\item Aritmatika Pengurangan
		\begin{verbatim}
			x=5
			y=3
			print(x-y)
		\end{verbatim}
	\subitem Output yang keluar dari source code tersebut adalah 2.

	\item Aritmatika Pengurangan
		\begin{verbatim}
			x=5
			y=3
			print(x*y)
		\end{verbatim}
	\subitem Output yang keluar dari source code tersebut adalah 15.

	\item Aritmatika Pembagian
		\begin{verbatim}
			x=5
			y=3
			print(x/y)
		\end{verbatim}
	\subitem Output yang keluar dari source code tersebut adalah 1,67.

	\item Merubah Integer Ke String
		\begin{verbatim}
			x=5
			y=3
			z=str(x)+str(y)
		\end{verbatim}

	\item Merubah String Ke Integer
		\begin{verbatim}
			x='5'
			y='3'
			z=int(x)+int(y)
		\end{verbatim}
\par
\item Python memiliki dua type pengulangan yaitu While Looping dan For Looping

\begin{itemize}
	\item Contoh While Looping
\end{itemize}
		\begin{verbatim}
			i = 1
			while i < 6:
		  	print(i)
 			 i += 1
		\end{verbatim}
	\subitem Output yang akan keluar adalah memunculkan angka 1 samapi 5 dikarenakan adanya source code
		\begin{verbatim}
			while i<6
		\end{verbatim}
	
	\item Contoh For Looping
		\begin{verbatim}
			for x in "banana":
			print(x)
		\end{verbatim}
	\subitem Output yang keluar adalah mengulang huruf b a n a n a secara vertikal.
\par
\item Contoh source code condition adalah seperti berikut
	\begin{verbatim}
		a = 33
		b = 33
		if b > a:
  		print("b is greater than a")
		elif a == b:
  		print("a and b are equal")
	\end{verbatim}
\subitem Output dari source code tersebut adalah a and b are equal di karenakan nilai vaiabel a dan b sama.

\par
\item Jenis Erorr yang ditemukan
\subitem dalam kasus ini erorr yang saya temukan adalah source code pada nomor aritmatika pembagian.
	\begin{verbatim}
		x=5
		y='3'
		print(x/y)
	\end{verbatim}
\subitem pada source code tersebut akan terjadi erorr dikarenakan integer dan string tidak dapat di persatukan. Untu penyelesaiannya adalah sebagai berikut.
	\begin{verbatim}
		x=5
		y=3
		print(x/y)
	\end{verbatim}
\par
	\begin{verbatim}
		x = 3
		try:
		print(x)
		except NameError:
		print("Variable x is not defined")
		except:
		print("Something else went wrong")
	\end{verbatim}
\subitem Output yang dikeluarkan adalah 3 dikarenakan variabel dari x bernilai 3.
\end{enumerate}




\chapter{Judul Bagian Kedua}
\section{Rangga Putra Ramdhani}
\subsubsection{Pemahanan Teori}
\begin{enumerate}
    \item Apa itu fungsi, inputan fungsi dan kembalian fungsi dengan contoh kode program
    lainnya.
    Fungsi adalah bagian dari program yang dapat digunakan ulang.
    Berikut merupakan contoh fungsi dan cara pemanggilannya
    \lstinputlisting[firstline=124, lastline=127]{src/1174056_praktek.py}

    Fungsi dapat membaca parameter, parameter adalah nilai yang disediakan kepada fungsi, dimana nilai ini akan menentukan output yang akan dihasilkan fungsi.
    \lstinputlisting[firstline=129, lastline=132]{src/1174056_praktek.py}

    Statemen return digunakan untuk keluar dari fungsi. Kita juga dapat menspesifikasikan nilai kembalian.
    \lstinputlisting[firstline=134, lastline=141]{src/1174056_praktek.py}

    \item Apa itu paket dan cara pemanggilan paket atau library dengan contoh kode
    program lainnya.
    Untuk memudahkan dalam pemanggilan fungsi yang di butuhkan, agar dapat dipanggil berulang.
    Cara pemanggilannya
    \lstinputlisting[firstline=143, lastline=144]{src/1174056_praktek.py}

    \item Jelaskan Apa itu kelas, apa itu objek, apa itu atribut, apa itu method dan
    contoh kode program lainnya masing-masing.
    kelas merupakan sebuah blueprint yang mepresentasikan objek.
    objek adalah hasil cetakan dadri sebuah kelas.
    method adalah suatu upaya yang digunakan oleh object.
    \lstinputlisting[firstline=146, lastline=168]{src/1174056_praktek.py}

    \item Jelaskan cara pemanggikan library kelas dari instansiasi dan pemakaiannya den-
    gan contoh program lainnya.
    Cara Pemanggilanya 
    \begin{itemize}
        \item pertama import terlebih dahulu filenya.
        \item kemudian buat variabel untuk menampung datanya
        \item setelah itu panggil nama classnya dan panggil methodnya
        \item Gunakan perintah print untuk menampilkan hasilnya

    \end{itemize}
    \lstinputlisting[firstline=170, lastline=175]{src/1174056_praktek.py}

    \item Jelaskan dengan contoh pemakaian paket dengan perintah from kalkulator im-
    port Penambahan disertai dengan contoh kode lainnya.
    Penggunaan paket from namafile import, itu berfungsi untuk memanggil file dan fungsinya
    \lstinputlisting[firstline=143, lastline=144]{src/1174056_praktek.py}

    \item Jelaskan dengan contoh kodenya, pemakaian paket fungsi apabila le library
    ada di dalam folder.
    Pemakaian paket adalah perkumpulan fungsi-fungsi. contoh kodenya adalah sebagai berikut :

    \item Jelaskan dengan contoh kodenya, pemakaian paket kelas apabila le library ada
    di dalam folder.
    \lstinputlisting[firstline=184, lastline=184]{src/1174056_praktek.py}

\end{enumerate}
\subsubsection{Ketrampilan Pemrograman}
\begin{enumerate}
    \item Buatlah fungsi dengan inputan variabel NPM, dan melakukan print luaran huruf
    yang dirangkai dari tanda bintang, pagar atau plus dari NPM kita. Tanda
    bintang untuk NPM mod 3=0, tanda pagar untuk NPM mod 3 =1, tanda plus
    untuk NPM mod3=2.
    \lstinputlisting[firstline=184, lastline=234]{src/1174056_praktek.py}

    \item Buatlah fungsi dengan inputan variabel berupa NPM. kemudian dengan meng-
    gunakan perulangan mengeluarkan print output sebanyak dua dijit belakang
    NPM.
    \lstinputlisting[firstline=237, lastline=243]{src/1174056_praktek.py}

    \item Buatlah fungsi dengan dengan input variabel string bernama NPM dan beri
    luaran output dengan perulangan berupa tiga karakter belakang dari NPM se-
    banyak penjumlahan tiga dijit tersebut.
    \lstinputlisting[firstline=245, lastline=255]{src/1174056_praktek.py}

    \item Buatlah fungsi hello word dengan input variabel string bernama NPM dan
    beri luaran output berupa digit ketiga dari belakang dari variabel NPM meng-
    gunakan akses langsung manipulasi string pada baris ketiga dari variabel NPM.
    \lstinputlisting[firstline=257, lastline=263]{src/1174056_praktek.py}

    \item buat fungsi program dengan input variabel NPM dan melakukan print nomor npm satu persatu kebawah.
    \lstinputlisting[firstline=265, lastline=269]{src/1174056_praktek.py}

    \item Buatlah fungsi dengan inputan variabel NPM, didalamnya melakukan penjum-
    lahan dari seluruh dijit NPM tersebut, wajib menggunakan perulangan dan
    atau kondisi.
    \lstinputlisting[firstline=272, lastline=279]{src/1174056_praktek.py}

    \item Buatlah fungsi dengan inputan variabel NPM, didalamnya melakukan melakukan
    perkalian dari seluruh dijit NPM tersebut, wajib menggunakan perulangan dan
    atau kondisi.
    \lstinputlisting[firstline=281, lastline=288]{src/1174056_praktek.py}

    \item Buatlah fungsi dengan inputan variabel NPM, Lakukan print NPM anda tapi
    hanya dijit genap saja. wajib menggunakan perulangan dan atau kondisi.
    \lstinputlisting[firstline=290, lastline=296]{src/1174056_praktek.py}

    \item Buatlah fungsi dengan inputan variabel NPM, Lakukan print NPM anda tapi
    hanya dijit ganjil saja. wajib menggunakan perulangan dan atau kondisi.
    \lstinputlisting[firstline=298, lastline=304]{src/1174056_praktek.py}

    \item Buatlah fungsi dengan inputan variabel NPM, Lakukan print NPM anda tapi
    hanya dijit yang termasuk bilangan prima saja. wajib menggunakan perulangan
    dan atau kondisi.
    \lstinputlisting[firstline=306, lastline=320]{src/1174056_praktek.py}

    \item Buatlah satu library yang berisi fungsi-fungsi dari nomor diatas dengan nama
    le rangga.py dan berikan contoh cara pemanggilannya pada le main.py.
    \lstinputlisting[firstline=7, lastline=7]{src/mainn.py}

    \item Buatlah satu library class dengan nama le kelas3lib.py yang merupakan mod-
    ikasi dari fungsi-fungsi nomor diatas dan berikan contoh cara pemanggilannya
    pada le mainn.py.
    \lstinputlisting[firstline=8, lastline=9]{src/mainn.py}
    
\end{enumerate}
\subsubsection{Ketrampilan Penanganan Error}
Error yang di dapat dari mengerjakan tugas ini adalah type error, cara menaggulaginya dengan cara mengecheck kembali codingannya
kemudian run kembali aplikasinya
berikut contoh Penggunaan fungsi try dan exception
\lstinputlisting[firstline=177, lastline=182]{src/1174056_praktek.py}

\section{Yusniar Nur Syarif Sidiq/1164089}
\subsection{Teori}
\begin{enumerate}
\item Fungsi merupakan sebuah bagian dari program yang dapat digunakan ulang dan memiliki inputan variabel serta nilai yang akan di kembalikannya. Contohnya adalah source code berikut ini,
	 \lstinputlisting{src/chapter2/1164089/1164089_1.py}
Dalam dalam source code tersebut akan mengeluarkan output Hallo 1164089 ketika kita running di dalam spyder.

\item Library dalam python disini merupakan kumpulan dari fungsi dan cara pemanggilannya adalah dengan melakukan import file librarynya. Sebagai contoh, buatlah Matematika.py dan 1164089\_2.py, simpan dalam satu folder. Untuk Matematika.py isikan fungsi sebagai berikut
	 \lstinputlisting{src/chapter2/1164089/Matematika.py}
Untuk memanggil fungsi tersebut adalah dengan melakukan import  Matematika.py pada 1164089\_2.py adalah sebagai berikut,
	 \lstinputlisting{src/chapter2/1164089/1164089_2.py}

\item Class merupakan salah satu cara untuk membuat sebuah kode yang mempunyai objek serta atribut tertentu sehingga akan lebih mudah dalam mengorganisasi berbagai fungsi dan statenya. Objek disini merupakan instansiasi atau perwujudan dari sebuah class. Untuk membuat class yang memiliki objek serta atribut dapat dilihat pada source code berikut ini, dimana kita akan membaut file bernama mtk.py
	\lstinputlisting{src/chapter2/1164089/mtk.py}
Self tersebut berfungsi untuk menunjukkan variabel lokal dari class tersebut. Untuk memanggil class tersebut kita akan membuat file bernama 1164089\_3.py dan kita akan melakukan import mtk.py pada file tersebut, untuk source codenya dapat dilihat seperti berikut,
	\lstinputlisting{src/chapter2/1164089/1164089_3.py}

\item Cara memanggilnya yaitu
	\begin{itemize}
		\item Pertama import terlebih dahulu filenya
		\item Buat variabel yang berfungsi menampung data
		\item Panggil nama classnya dan methodnya
		\item Gunakan perintah print untuk menampilkannya
	\end{itemize}
Sebagai contoh perhatikan source code ini,
	\lstinputlisting{src/chapter2/1164089/1164089_4.py}

\item Dimana kita akan melakukan membuka library Matematika.py dan akan melakukan import dari fungsi di dalamnya yaitu matematika, sehingga akan lebih simple dalam penulisan source codenya adalah sebagai berikut,
	\lstinputlisting{src/chapter2/1164089/1164089_5.py}

\item

\item 
\end{enumerate}

\subsection{Keterampilan Pemrograman}
\begin{enumerate}

\item Soal No 1 \lstinputlisting{src/chapter2/1164089/1164089_21.py}

\item Soal No 2 \lstinputlisting{src/chapter2/1164089/1164089_22.py}

\item Soal No 3 \lstinputlisting{src/chapter2/1164089/1164089_23.py}

\item Soal No 4 \lstinputlisting{src/chapter2/1164089/1164089_24.py}

\item Soal No 5 \lstinputlisting{src/chapter2/1164089/1164089_25.py}

\item Soal No 6 \lstinputlisting{src/chapter2/1164089/1164089_26.py}

\item Soal No 7 \lstinputlisting{src/chapter2/1164089/1164089_27.py}

\item Soal No 8 \lstinputlisting{src/chapter2/1164089/1164089_28.py}

\item Soal No 9 \lstinputlisting{src/chapter2/1164089/1164089_29.py}

\item Soal No 10 \lstinputlisting{src/chapter2/1164089/1164089_30.py}

\item Soal No 11 \lstinputlisting{src/chapter2/1164089/1164089_31.py}

\item Soal No 12 \lstinputlisting{src/chapter2/1164089/1164089_32.py}
\end{enumerate}

\subsection{Penanganan Erorr}
\begin{enumerate}

\item Erorr yang saya temui di antaranya adalah Systax Erorr, dimana suatu keadaan script python mengalami kesalahan dalam penulisannya dan solusi dari permasalahan ini adalah dengan memperbaiki script penulisan yang salah. Untuk contoh fungsi trx except dapat dilihat pada source code berikut ini,

	\lstinputlisting{src/chapter2/1164089/1164089_33.py}

\end{enumerate}



\bibliographystyle{IEEEtran} 
%\def\bibfont{\normalsize}
\bibliography{references}


%%%%%%%%%%%%%%%
%%  The default LaTeX Index
%%  Don't need to add any commands before \begin{document}
\printindex

%%%% Making an index
%% 
%% 1. Make index entries, don't leave any spaces so that they
%% will be sorted correctly.
%% 
%% \index{term}
%% \index{term!subterm}
%% \index{term!subterm!subsubterm}
%% 
%% 2. Run LaTeX several times to produce <filename>.idx
%% 
%% 3. On command line, type  makeindx <filename> which
%% will produce <filename>.ind 
%% 
%% 4. Type \printindex to make the index appear in your book.
%% 
%% 5. If you would like to edit <filename>.ind 
%% you may do so. See docs.pdf for more information.
%% 
%%%%%%%%%%%%%%%%%%%%%%%%%%%%%%

%%%%%%%%%%%%%% Making Multiple Indices %%%%%%%%%%%%%%%%
%% 1. 
%% \usepackage{multind}
%% \makeindex{book}
%% \makeindex{authors}
%% \begin{document}
%% 
%% 2.
%% % add index terms to your book, ie,
%% \index{book}{A term to go to the topic index}
%% \index{authors}{Put this author in the author index}
%% 
%% \index{book}{Cows}
%% \index{book}{Cows!Jersey}
%% \index{book}{Cows!Jersey!Brown}
%% 
%% \index{author}{Douglas Adams}
%% \index{author}{Boethius}
%% \index{author}{Mark Twain}
%% 
%% 3. On command line type 
%% makeindex topic 
%% makeindex authors
%% 
%% 4.
%% this is a Wiley command to make the indices print:
%% \multiprintindex{book}{Topic index}
%% \multiprintindex{authors}{Author index}

\end{document}
